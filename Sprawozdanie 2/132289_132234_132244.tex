\documentclass[polish,polish,a4paper]{article}
\usepackage{circuitikz}
\usepackage{tikz}
\usepackage{verbatim}
\usepackage{amsmath}
\usepackage{amssymb,amsfonts,amsthm}
\usepackage[T1]{fontenc}
\usepackage[utf8]{inputenc}
\usepackage[polish]{babel}
\usepackage{pslatex}
\usepackage{pgfplots}
\usepackage{anysize}
\marginsize{2.5cm}{2.5cm}{3cm}{3cm}
\usepackage{xcolor,colortbl} 
\usepackage{setspace}
\usepackage{float}
\usepackage{makecell}
\newcolumntype{|}{!{\vrule width 1pt}}
\newcommand{\PRzFieldDsc}[1]{\sffamily\bfseries\scriptsize #1}
\newcommand{\PRzFieldCnt}[1]{\textit{#1}}
\newcommand{\PRzHeading}[8]{
%% #1 - nazwa laboratorium
%% #2 - kierunek 
%% #3 - specjalność 
%% #4 - rok studiów 
%% #5 - symbol grupy lab.
%% #6 - temat 
%% #7 - numer lab.
%% #8 - skład grupy ćwiczeniowej

\begin{center}
\begin{tabular}{ p{0.32\textwidth} p{0.15\textwidth} p{0.15\textwidth} p{0.12\textwidth} p{0.12\textwidth} }

  &   &   &   &   \\
\Xhline{0.8pt}
\multicolumn{5}{|c|}{}\\[-1ex]
\multicolumn{5}{|c|}{{\LARGE #1}}\\
\multicolumn{5}{|c|}{}\\[-1ex]

\Xhline{0.8pt}
\multicolumn{1}{|l|}{\PRzFieldDsc{Kierunek}}	& \multicolumn{1}{|l|}{\PRzFieldDsc{Specjalność}}	& \multicolumn{1}{|l|}{\PRzFieldDsc{Rok studiów}}	& \multicolumn{2}{|l|}{\PRzFieldDsc{Symbol grupy lab.}} \\
\multicolumn{1}{|c|}{\PRzFieldCnt{#2}}		& \multicolumn{1}{|c|}{\PRzFieldCnt{#3}}		& \multicolumn{1}{|c|}{\PRzFieldCnt{#4}}		& \multicolumn{2}{|c|}{\PRzFieldCnt{#5}} \\

\Xhline{0.8pt}
\multicolumn{4}{|l|}{\PRzFieldDsc{Temat Laboratorium}}		& \multicolumn{1}{|l|}{\PRzFieldDsc{Numer lab.}} \\
\multicolumn{4}{|c|}{\PRzFieldCnt{#6}}				& \multicolumn{1}{|c|}{\PRzFieldCnt{#7}} \\

\Xhline{0.8pt}
\multicolumn{5}{|l|}{\PRzFieldDsc{Skład grupy ćwiczeniowej oraz numery indeksów}}\\
\multicolumn{5}{|c|}{\PRzFieldCnt{#8}}\\

\Xhline{0.8pt}
\multicolumn{3}{|l|}{\PRzFieldDsc{Uwagi}}	& \multicolumn{2}{|l|}{\PRzFieldDsc{Ocena}} \\
\multicolumn{3}{|c|}{\PRzFieldCnt{\ }}		& \multicolumn{2}{|c|}{\PRzFieldCnt{\ }} \\

\Xhline{0.8pt}
\end{tabular}
\end{center}
}


\definecolor{Yellow}{rgb}{1,1,0.8}
\definecolor{Steel}{rgb}{0.9,0.9,1}
\newcolumntype{y}{>{\columncolor{Yellow}}l}
\newcommand{\rowstyle}[1]{\gdef\currentrowstyle{#1}%
#1\ignorespaces
}


\begin{document}

\PRzHeading{Podstawy Elektroniki}{Informatyka}{--}{II}{I1}{Twierdzenie Thevenina}{1}{Artur Mostowski (132289), Łukasz Grygier(132234), Tomasz Jurek(132244)}{}
\section{}%1
\section{}%2
\begin{table}[!h]
\centering
\begin{tabular}{|y|l|l|l|l|}
\Xhline{0.8pt}
\rowcolor{Yellow}
\textbf{Lp.}
& 
\textbf{R}
&
\textbf{Kod paskowy (KP)}
&
\textbf{Wartość odczytana z KP}
&
\textbf{Wartość rezystancji zmierzona}
\\
\Xhline{0.8pt}
\textbf{1}
& 
\textbf{R1}
&
{Brązowy,Czarny,Brązowy,Złoty}
&
{$100 \Omega \pm 5\%$}
&
{$97,86 \Omega$}
\\
\Xhline{0.8pt}\rowcolor{Steel}\cellcolor{Yellow}\textbf{2}
& 
\textbf{R2}
&

{Zielony,Brązowy,Brązowy,Złoty}
&
{$510 \Omega \pm 5\%$}
&
{$499,53 \Omega$}
\\
\Xhline{0.8pt}
\textbf{3}
& 
\textbf{R3}
&
{Czerwony,Czerwony,Brązowy,Złoty}
&
{$220 \Omega \pm 5\%$}
&
{$215,57 \Omega$}
\\
\Xhline{0.8pt}\rowcolor{Steel}\cellcolor{Yellow}\textbf{4}
& 
\textbf{R4}
&
{Brązowy,Czarny,Brązowy,Złoty}
&
{$100 \Omega \pm 5\%$}
&
{$97,85 \Omega$}
\\
\Xhline{0.8pt}
\end{tabular}
\caption{Odczytania wartości rezystancji na podstawie kodów paskowych używanych rezystorów wraz z wynikami pomiarów ich faktycznej wartości }
\end{table}
\section{}%3
\begin{table}[!h]
\centering
\begin{tabular}{|y|l|l|}
\Xhline{0.8pt}
\rowcolor{Yellow}
\textbf{Lp.}
& 
\textbf{U\textsubscript{th}}
&
\textbf{R\textsubscript{th}}
\\
\Xhline{0.8pt}
\textbf{1}
& 

&

\\
\Xhline{0.8pt}\rowcolor{Steel}\cellcolor{Yellow}\textbf{2}
& 

&

\\
\Xhline{0.8pt}
\textbf{3}
& 

&

\\
\Xhline{0.8pt}\rowcolor{Steel}\cellcolor{Yellow}\textbf{4}
& 

&

\\
\Xhline{0.8pt}
\end{tabular}
\caption{Odczytania wartości rezystancji na podstawie kodów paskowych używanych rezystorów wraz z wynikami pomiarów ich faktycznej wartości }
\end{table}
\section{}%4

\begin{figure}[H]
\begin{center}
\begin{circuitikz} \draw
(0,0) to[american voltage source, l=$V_1$,a=5V]  (0,-1) to (0,-3) to (3,-3) to[R,l=$R_3$,a=220$\Omega$] (3,-1) to[R,l=$R_2$,a=510$\Omega$] (3,1) to[R,l=$R_4$,a=100$\Omega$] (0,1) to (0,0)
(3,1)to(6,1)to[R,l=$R_1$,a=100$\Omega$](6,-1)to(3,-1)
;
\end{circuitikz}
\end{center}
\end{figure}
\section{}%5
\begin{table}[!h]
\centering
\begin{tabular}{|y|l|l|l|}
\Xhline{0.8pt}
\rowcolor{Yellow}
\textbf{Lp.}
& 
\textbf{U\textsubscript{th}}
&
\textbf{R\textsubscript{th}}
&
\textbf{I\_R\textsubscript{x}}
\\
\Xhline{0.8pt}
\textbf{1}
& 

&

&

\\
\Xhline{0.8pt}\rowcolor{Steel}\cellcolor{Yellow}\textbf{2}
& 

&

&

\\
\Xhline{0.8pt}
\textbf{3}
& 

&

&

\\
\Xhline{0.8pt}
\end{tabular}
\caption{Odczytania wartości rezystancji na podstawie kodów paskowych używanych rezystorów wraz z wynikami pomiarów ich faktycznej wartości }
\end{table}
\section{}%6
\section{}%7
\begin{table}[!h]
\centering
\begin{tabular}{|y|l|l|}
\Xhline{0.8pt}
\rowcolor{Yellow}
\textbf{Lp.}
& 
\textbf{I\_R\textsubscript{x}(z tw.Thevenina}
&
\textbf{I\_R\textsubscript{x}(z pomiarów}
\\
\Xhline{0.8pt}
\textbf{1}
& 

&

\\
\Xhline{0.8pt}\rowcolor{Steel}\cellcolor{Yellow}\textbf{2}
& 

&

\\
\Xhline{0.8pt}
\textbf{3}
& 

&

\\
\Xhline{0.8pt}\rowcolor{Steel}\cellcolor{Yellow}\textbf{4}
& 

&

\\
\Xhline{0.8pt}
\end{tabular}
\caption{Odczytania wartości rezystancji na podstawie kodów paskowych używanych rezystorów wraz z wynikami pomiarów ich faktycznej wartości }
\end{table}
\section{}%8
\section{Zadanie A}
\subsection{Część I}
\begin{table}[!h]
\centering
\begin{tabular}{|y|l|l|l|}
\hline
\rowcolor{Yellow}
\textbf{\underline{R}} & \textbf{\underline{Barwy}} & \textbf{\underline{Odczyt}} &\textbf{\underline{Pomiar}} \\
\hline
\textbf{R1} & Czerwony,Czarny,Zielony,Złoty & 2M$\Omega \pm 5\%$ &
1,99M$\Omega$
\\
\hline
\textbf{R2} & Brązowy,Czarny,Brązowy,Złoty & 100$\Omega \pm 5\%$ &
98,78$\Omega$
\\
\hline
\textbf{R3} & Czerwony,Czarny,Czerwony,Złoty & 2k$\Omega \pm 5\%$ &
1964,1$\Omega$
\\
\hline
\textbf{R4} & Brązowy,Czarny,Czerwony,Złoty & 1k$\Omega \pm 5\%$ &
987,1$\Omega$
\\
\hline
\textbf{R5} & Szary,Czerwony,Brązowy,Złoty & 820$\Omega \pm 5\%$ &
811,2$\Omega$
\\
\hline
\textbf{R6} & Czerwony,Czarny,Brązowy,Złoty & 200$\Omega \pm 5\%$ &
197,3$\Omega$
\\
\hline

\end{tabular}
\caption{Odczytania wartości rezystancji na podstawie kodów paskowych rezystorów wraz z wynikami pomiarów ich faktycznej wartości przy pomocy multimetru RIGOL DS1022}
\end{table}

\subsection{Część II}
\begin{table}[!h]
\centering
\begin{tabular}{|y|l|l|l|y|l|l|l|}
\hline
\rowcolor{Yellow}
\textbf{\underline{C}} & \textbf{\underline{Oznaczenie}} & \textbf{\underline{Odczyt}} & \textbf{\underline{Pomiar}} & \textbf{\underline{C}} & \textbf{\underline{Oznaczenie}} & \textbf{\underline{Odczyt}} & \textbf{\underline{Pomiar}} \\
\hline
\textbf{C1} & 271J & 270pF & 249pF &
\textbf{C4} & 10 $\mu$ & 10 $\mu$ & 10,66 $\mu$ \\
\hline
\textbf{C2} & 222 & 2,2nF & 1,9nF &
\textbf{C5} & 22 $\mu$ & 22 $\mu$ & 22,463 $\mu$ \\
\hline
\textbf{C3} & 472AEC & 4,7nF & 4,5nF &
\textbf{C6} & 47 $\mu$ & 47 $\mu$ & 43,879 $\mu$ \\
\hline

\end{tabular}
\caption{Odczytania wartości pojemności kondensatorów na podstawie ich oznaczeńwraz z wynikami pomiarów ich faktycznej wartości przy pomocy mostka pomiarowego}
\end{table}
\subsection{Część III}
\begin{table}[!h]
\centering
\begin{tabular}{|y|l|}
\hline
\rowcolor{Yellow}
\textbf{\underline{L}} & \textbf{\underline{Pomiar}} \\
\hline
\textbf{L1} &
29,7mH
\\
\hline
\textbf{L2}&
30,7mH 
\\
\hline
\textbf{L3} &
33,5mH
\\
\hline

\end{tabular}
\caption{Wyniki pomiaru indukcyjności wybranych cewek przy pomocy mostka pomiarowego}
\end{table}
\section{Zadanie B}
\subsection{Część I}
\begin{figure}[h]
\begin{center}
\begin{circuitikz} \draw
(-1,0) to (-5,0) to (-5,2) to[R=$R_7$,a=1k$\Omega$,-*] (-3,2)
(-1,0)node[right]{A}
(-3,2) to (-3,1) to[R=$R_6$,a=200$\Omega$] (-1,1) to[short,-*] (-1,2)
(-3,2) to (-3,3) to[R=$R_5$,a=100$\Omega$] (-1,3) to (-1,2)
(-1,2) to[short,-*] (0,2) to (0,1) to[short,-*] (0,3) to[short,-*] (0,5) to (0,7)
(0,1) to[R, l=$R_4$,a=270$\Omega$] (2,1)
(0,3) to[R=$R_3$,a=1k$\Omega$] (2,3) to[short,-*] (2,3)
(0,5) to[R=$R_2$,a=3k$\Omega$] (2,5) to[short,-*] (2,5)
(0,7) to[R=$R_1$,a=2k$\Omega$] (2,7)
(2,1) to[short,-*] (2,2) to (2,7)
(2,2) to[short,-*] (3,2) to (3,3) to (3,1)
(5,2) to (5,3) to (5,1)
(3,1) to[R=$R_9$,a=100$\Omega$] (5,1)
(3,3) to[R=$R_8$,a=1$\Omega$] (5,3)
(2,0) to (6,0) to (6,2) to[short,-*] (5,2)
(2,0)node[left]{B}
;
\end{circuitikz}
\end{center}
\end{figure}



Oporniki $R_1$, $R_2$, $R_3$ oraz $R_4$ są połączone równolegle, więc opór zastęgloriouse pc master racezy tych oporników można policzyć
w następujący sposób:
\begin{equation}
\begin{aligned}
 R_{1234} &= \frac{R_{1} \cdot R_{2} \cdot R_{3} \cdot R_{4}}{R_{1} \cdot R_{2} \cdot R_{3}+R_{1} \cdot R_{2} \cdot R_{4}+
R_{1} \cdot R_{3} \cdot R_{4}+R_{2} \cdot R_{3} \cdot R_{4}}
  \end{aligned} 
\end{equation}

W ten sam sposób liczymy opór zastępczy oporników $R_5$ i $R_6$:

\begin{equation}
\begin{aligned}
R_{56} &= \frac{R_5 \cdot R_6}{R_{5}+R_{6}}
  \end{aligned} 
\end{equation}

Oraz oporników $R_8$ i $R_9$:

\begin{equation}
\begin{aligned}
R_{89} &= \frac{R_{8} \cdot R_{9}}{R_{8}+R_{9}}
  \end{aligned}
\end{equation}

Po tym możemy policzyć opór zastępczy całego obwodu:

\begin{equation}
\begin{aligned}
R_z &= R_{1234} + R_{56} + R_{89} + R_7
\\
\\
R_{z} &= \frac{R_{1} \cdot R_{2} \cdot R_{3} \cdot R_{4}}{R_{1} \cdot R_{2} \cdot R_{3}+R_{1} \cdot R_{2} \cdot R_{4}+R_{1} \cdot R_{3} \cdot R_{4}+R_{2} \cdot R_{3} \cdot R_{4}} + \frac{R_5 \cdot R_6}{R_{5}+R_{6}} +  \frac{R_{8} \cdot R_{9}}{R_{8}+R_{9}} + R_7
\\
\\
 R_z &= \frac{2000\Omega  \cdot  3000\Omega  \cdot  1000\Omega  \cdot  270\Omega}{2000\Omega  \cdot  3000\Omega  \cdot  1000\Omega + 2000\Omega  \cdot  1000\Omega  \cdot  270\Omega + 2000\Omega  \cdot  3000\Omega  \cdot  270\Omega + 3000\Omega  \cdot  1000\Omega  \cdot  270\Omega} 
\\
\\
&+ \frac{100\Omega  \cdot  200\Omega}{100\Omega+200\Omega} + \frac{1\Omega  \cdot  100\Omega}{1\Omega + 100\Omega} + 1000\Omega
\\
\\
R_{z} &= 180,6\Omega + 66,67\Omega + 0,99\Omega + 1000\Omega = 1248,26\Omega
  \end{aligned} 
\end{equation}


\subsection{Część II}
\subsubsection{Obwód 1}
\begin{figure}[H]
\begin{center}
 \begin{circuitikz} \draw
 
 (0,0) to [short,-*] (2,0) to[R=$R_1$,a=1k$\Omega$,-*] (2,-2) to (0,-2)
 (0,0)node[left]{A}
 (0,-2)node[left]{B}
 (2,0) to[R=$R_2$,a=2k$\Omega$] (4,0) to (4,-2)
 (2,-2) to[R=$R_3$,a=2k$\Omega$] (4,-2)
 ;
\end{circuitikz}
\end{center}
\end{figure}

\setcounter{equation}{0}

\begin{equation}
\begin{aligned}
R_{23} &= R_2 + R_3
\\
\\
R_z &= \frac{R_{23}  \cdot  R_1}{R_{23}+R_1}
\\
\\
R_z &= \frac{4000\Omega  \cdot  1000\Omega}{4000\Omega + 1000\Omega} = 800\Omega
  \end{aligned} 
\end{equation}
Wynik Pomiaru:$798,653\Omega$
\subsubsection{Obwód 2}
\begin{figure}[H]
\begin{center}
\begin{circuitikz} \draw
(-1,0) to[short,-*] (-1,1)
(-1,0) node[left]{A}
(1,0) to (1,1)
(1,0) node[right]{B}
(-1,1) to[R=$R_4$,a=100$\Omega$,-*] (1,1)
(1,1) to (2,1)
(-1,1) to[short,-*] (-2,1) to (-5,1)
(-2,1) to[R,l=$R_2$,a=2k$\Omega$,-*] (-2,3)
(-5,1) to[R,l=$R_1$,a=1k$\Omega$] (-5,3)
(-1,3) to (-2,3) to (-5,3)
(-1,3) to[R=$R_3$,a=1k$\Omega$] (1,3) to (2,3)
(2,1) to[R,l=$R_5$,a=100$\Omega$] (2,3)
;
\end{circuitikz}
\end{center}
\end{figure}

\begin{equation}
\begin{aligned}
R_{12} &= \frac{R_1  \cdot  R_2}{R_1 + R_2}
\\
\\
R_{1235} &= R_{12} + R_3 + R_5
\\
\\
R_z &= \frac{R_{1235} \cdot R_4}{R_{1235}+R_4}
\\
\\
R_z &= \frac{ \frac{R_1  \cdot  R_2  \cdot  R_4 }{R_1 + R_2}+ R_3 \cdot R_4 + R_5 \cdot R_4}{\frac{R_1  \cdot  R_2}{R_1 + R_2} + R_3 + R_5 + R_4}
\\
\\
R_z &= \frac{ \frac{1000\Omega  \cdot  2000\Omega  \cdot  100\Omega }{1000\Omega + 2000\Omega}+ 1000\Omega \cdot 100\Omega + 100\Omega \cdot 100\Omega}{\frac{1000\Omega  \cdot  2000\Omega}{1000\Omega + 2000\Omega} + 1000\Omega + 100\Omega + 100\Omega}
\\
\\
R_z &= \frac{176666,67\Omega}{1866,67\Omega} = 94,64\Omega
  \end{aligned} 
\end{equation}
Wynik Pomiaru:$94,6\Omega$
\subsubsection{Obwód 3}
\begin{figure}[H]
\begin{center}
\begin{circuitikz} \draw
(0,0) to (1,0) to[short,-*] (2,0) to (2,2)
(2,2) to[R,l=$R_4$,a=1k$\Omega$] (4,2)
(2,0) to[R,l=$R_3$,a=100$\Omega$,-*] (4,0)
(4,2) to (4,0) to (5,0) to (6,0)
(5,0) to[R,l=$R_2$,a=2k$\Omega$,*-] (5,-2) to (6,-2)
(0,-2) to (1,-2)
(0,0) node[left]{A}
(0,-2) node[left]{B}
(1,0) to[R,l=$R_1$,a=2k$\Omega$,*-] (1,-2)

;
\end{circuitikz}
\end{center}
\end{figure}


\begin{equation}
R_z=2000\Omega
Pozostałe oporniki nie mając wpływu na opór, poniewaz prąd nie może przez nie płynąć od a do b.
\end{equation}
Wynik Pomiaru:$1984,84\Omega$
\subsubsection{Obwód 4}
\begin{figure}[H]
\begin{center}
\begin{circuitikz} \draw
(0,2) to[short,-*] (-1,2) to[short,-*] (-1,3) to[R,l=$R_2$,a=2k$\Omega$,-*] (-3,3) to (-3,2)
(0,2) node[right]{B}
(0,0) to (-1,0) to (-1,0) to (-3,0)
(0,0) node[right]{A}
(-1,2) to[R,l=$R_4$,a=1k$\Omega$,-*] (-1,0)
(-3,2) to[R,l=$R_3$,a=2k$\Omega$] (-3,0)
(-3,3) to[R,l=$R_1$,a=1k$\Omega$] (-5,3)
(1,3) to[R,l=$R_5$,a=100$\Omega$] (-1,3)
(-5,3) to (-5,4) to (1,4) to (1,3)
;
\end{circuitikz}
\end{center}
\end{figure}

\begin{equation}
\begin{aligned}
R_{15} &= R_1 + R_5
\\
\\
R_{125} &= \frac{R_{15}  \cdot  R_2}{R_{15} + R_2}
\\
\\
R_{1235} &= R_{125}+R_3
\\
\\
R_z &= \frac{R_{1235} \cdot R_4}{R_{1235}+R_4}
\\
\\
R_z &= \frac{ \frac{R_1  \cdot  R_2  \cdot  R_4 +R_5  \cdot  R_2  \cdot  R_4 }{R_1 + R_2 + R_5} + R_3 \cdot R_4}{\frac{R_1  \cdot  R_2 +R_5  \cdot  R_2 }{R_1 + R_2 + R_5} + R_3 + R_4}
\\
\\
R_z &= \frac{ \frac{ 1000\Omega  \cdot  2000\Omega  \cdot  1000\Omega + 100\Omega  \cdot  2000\Omega  \cdot  1000\Omega }{1000\Omega + 2000\Omega + 100\Omega} +2000\Omega \cdot 1000\Omega }{\frac{ 1000\Omega  \cdot  2000\Omega + 100\Omega  \cdot  2000\Omega }{1000\Omega + 2000\Omega + 100\Omega} +2000\Omega + 1000\Omega}
\\
\\
R_z &= \frac{ 2709677,4\Omega}{3709,6774 \Omega} =  730,43478  \Omega
  \end{aligned} 
\end{equation}
Wynik Pomiaru:$714,6\Omega$
\subsubsection{Obwód 5}
\begin{figure}[H]
\begin{center}
\begin{circuitikz} \draw
(0,0) to (1,0) to[short,-*] (2,0) to (2,2)
(2,2) to[R,l=$R_4$,a=1k$\Omega$] (4,2)
(2,0) to[R,l=$R_3$,a=2k$\Omega$,-*] (4,0)
(4,2) to (4,0) to[short,-*] (5,0) to (6,0)
(5,0) to[R,l=$R_2$,a=2k$\Omega$,-*] (5,-2) to (6,-2)
(0,-2) to[short,-*] (1,-2)
(0,0) node[left]{A}
(0,-2) node[left]{B}
(1,0) to[R,l=$R_1$,a=2k$\Omega$,*-] (1,-2)
(1,-2) to (5,-2)

;
\end{circuitikz}
\end{center}
\end{figure}
\begin{equation}
\begin{aligned}
R_{34} &= \frac{R_3  \cdot  R_4}{R_3 + R_4}
\\
\\
R_{234} &= R_{34}+R_2
\\
\\
R_z &= \frac{R_{234} \cdot R_1}{R_{234}+R_1}
\\
\\
R_z &= \frac{ \frac{R_3  \cdot  R_4  \cdot  R_1}{R_3 + R_4} + R_2 \cdot R_1}{\frac{R_3  \cdot  R_4}{R_3 + R_4} + R_2 + R_1}
\\
\\
R_z &= \frac{ \frac{ 2000\Omega  \cdot  1000\Omega  \cdot  2000\Omega }{2000\Omega + 1000\Omega} +2000\Omega \cdot 2000\Omega }{\frac{ 2000\Omega  \cdot  1000\Omega }{2000\Omega + 1000\Omega} +2000\Omega + 2000\Omega}
\\
\\
R_z &= \frac{ 5333333,3\Omega}{4666,6667 \Omega} =  1142,8571\Omega
  \end{aligned} 
\end{equation}
Wynik Pomiaru:$1138,68\Omega$
\subsubsection{Obwód 6}
\begin{figure}[H]
\begin{center}
\begin{circuitikz} \draw
(0.5,0) to (0,0) to (-3,0) to[R,l=$R_3$,a=1k$\Omega$,-*] (-3,-2) to (-4,-2) to (-2,-2)
(0.5,0) node[right]{A}
(-4,-2) to[R,l=$R_1$,a=100$\Omega$] (-4,-4) to[R,l=$R_5$,a=1k$\Omega$,*-] (-6,-4) to (-6,-6) to (0,-6) to (0,-3)
(0.5,-1) to (0,-1) to[short,-*] (0,-3) to[R,l=$R_4$,a=2k$\Omega$,-*]  (-2,-3) to (-2,-2) 
(-2,-3) to[R,l=$R_2$,a=2k$\Omega$] (-2,-5) to (-4,-5) to (-4,-4)
(0.5,-1) node[right]{B}
;
\end{circuitikz}
\end{center}
\end{figure}

\begin{equation}
\begin{aligned}
R_{12} &= \frac{R_1  \cdot  R_2}{R_1 + R_2}
\\
\\
R_{125} &= R_{12}+R_5
\\
\\
R_{1245} &= \frac{R_{125}  \cdot  R_4}{R_{125} + R_4}
\\
\\
R_z &= R_{1245}+R_3
\\
\\
R_z &= \frac{\frac{R_1  \cdot  R_2  \cdot  R_4}{R_1 + R_2}+R_5  \cdot  R_4}{\frac{R_1  \cdot  R_2}{R_1 + R_2}+R_5 + R_4}+R_3
\\
\\
R_z &= \frac{ \frac{ 100\Omega  \cdot  2000\Omega  \cdot  2000\Omega }{100\Omega + 2000\Omega}+1000\Omega  \cdot  2000\Omega }{\frac{ 100\Omega  \cdot  2000\Omega }{100\Omega + 2000\Omega}+1000\Omega + 2000\Omega}+1000\Omega
\\
\\
R_z &= \frac{ 2190476,2  \Omega}{3095,2381 \Omega} +1000\Omega=  1707,69231 \Omega
  \end{aligned} 
\end{equation}
Wynik Pomiaru:$1684,35\Omega$
\subsubsection{Skąd różnice?}
Różnice między oporami zastępczymi obliczonymi, a zmierzonymi wynikają zapewne z faktu, iż każdy użyty do zbudowania obwodu opornik ma własną tolerancję. Tolerancje te składają się na sumaryczny błąd pomiaru.

Na wyniki pomiaru może mieć także niewielki wpływ opór własny kabli użytych do połączenia obwodu ze źródłem prądu, który pomimo tego, że relatywnie mały, może wpłynąć na wyniki, szczególnie gdy badany obwód także ma stosunkowo niski opór zastępczy.
\section{Zadanie C}
\subsection{Część I}
\setcounter{equation}{0}
\begin{table}[!h]
\centering
\begin{tabular}{|y|l|l|}
\hline
\rowcolor{Yellow}
U[V] & Pomiar[V] & Odczyt[V] \\
\hline
1  & 1,11 & 1,0 \\
\hline
3 & 3,15 & 3,0 \\
\hline
4,5 & 4,68 & 4,5 \\
\hline
11  & 11,24 & 11,0 \\
\hline
13  & 13,2 & 13,0 \\
\hline
25  & 25,34 & 25,0 \\
\hline
28 & 28,36 & 28,0 \\
\hline
\end{tabular}
\caption{Wyniki pomariu napięcia z sekcji DC POWER SUPPLY zestawu laboratoryjnego DF 6911}
\end{table}

Różnice pomiędzy wartościami zmierzonymi, a dczytanymi, wynikają zapewne zoporu własnego kabli, który pomimo tego, że wynosi tylko około $1 \Omega$ to może spowodować błędy pomiaru, szczególnie, że takich kabli użyto w tym pomiarze aż czterech.

\subsection{Część II}

Wyprowadzenie wzoru opisującego dzielnik napięcia (na rysunku poniżej).\\

\begin{figure}[H]
\begin{center}
\begin{circuitikz} \draw
(0,0) to[american voltage source, l=$U_{in}$] (0,-1) to (0,-3) to (2,-3) to[R, l=$R_2$,*-] (2,-1) to[R, l=$R_1$, *-] (2,1) to (0,1) to (0,0)
(2,-1) to (4,-1) to (4,-1.5)
(2,-3) to (4,-3) to (4,-2.5)
(4,-1.7)node[anchor=north]{$U_{out}$}
;
\end{circuitikz}
\end{center}
\end{figure}
\begin{equation}
\begin{aligned}
U &= I*R
\\
\\
U_{we} &= I*(R_1 + R_2) 
\\
\\
U_{wy} &= I * R_2
\\
\\
I_1 &= \frac{U_{we}}{R_1 + R_2}  
\\
\\
I_2 &=\frac{U_{wy}}{R_2}
\\
\\
I_1 &= I_2
\\
\\
\frac{U_{we}}{R_1 + R_2} &= \frac{U_{wy}}{R_2}
\\
\\
U_{wy} &= \frac{U_{we}*R_2}{R_1+R_2}
  \end{aligned} 
\end{equation}


\begin{equation}
\begin{aligned}
U_{we} &= 15V 
\\
\\
U_{wy} &= 1,5V
\\
\\
1,5&=15*\frac{R_2}{R_1+R_2}
\\
\\
R_2&=0,1*R_1+0,1*R_2 
\\
\\
\frac{R_1}{R_2}&=9 
  \end{aligned} 
\end{equation}

W zaproponowanym przez nas dzielniku napięcia opowniki mają opory $R_1=900\Omega$ oraz $R_2=100\Omega$.

\subsection{Część III}
\begin{figure}[H]
\begin{center}
\begin{circuitikz} \draw
(0,0) to[american voltage source, l=$V_1$,a=5V] (0,2) to (1,2) to[R,l=$R_1$,a=2.1k$\Omega$] (3,2) to (4,2) to (4,0) to (0,0)
;
\end{circuitikz}
\end{center}
\end{figure}
Pomiar spadków napięcia na rezystorze R1 oraz natężenia prądu w obwodzie przy użyciu płytki 

prototypowej.
\begin{equation}
\begin{aligned}
U_{r1}&=5,033V
\\
\\
I&=2,427mA 
  \end{aligned} 
\end{equation}
\subsection{Część IV}
Sprawdzenie praw Kirchhoffa dla obwodu podanego poniżej.
\begin{figure}[!h]
\begin{center}
\begin{circuitikz} \draw
(0,0) to[american voltage source, l=$V_1$] (3,0) to[R,l=$R_4$,a=1k$\Omega$,-*] (3,-2) to[R,l=$R_3$,a=200$\Omega$] (0,-2)
(0,0) to[R,l=$R_1$,a=100$\Omega$,-*] (0,-2) to[R,l=$R_2$,a=2k$\Omega$] (0,-4) to (3,-4) to (3,-2) 
;
\end{circuitikz}
\end{center}
\end{figure}

\begin{equation}
\begin{aligned}
R_z&=100\Omega+1000\Omega+\frac{200\Omega \cdot 2000\Omega}{200\Omega+2000\Omega}=1281,81\Omega
\\
\\
R_z&=R_1+R_4+\frac{R_3 \cdot R_2}{R_3+R_2}
\\
\\
U_1&=\frac{R_1}{R_z}*U_{we}=0,39V  
\\
\\
I_1&=\frac{U_1}{R_1}=3,9mA
\\
\\
U_{23}&=\frac{R_{23}}{R_z}*U_{we}=0,7V 
\\
\\
I_{23}&=\frac{U_{23}}{R_{23}}=3,85mA
\\
\\
U_4&=\frac{R_4}{R_z}*U_{we}=3,9V 
\\
\\
I_4&=3,9mA
\\
\\
I_2&=\frac{U_2}{R_z}=3,5mA
\\
\\
I_3&=3,8mA
\\
\\
I&=3,9mA
\end{aligned} 
\end{equation}
\newpage
Pomiary spadków napięć na rezystorach oraz prądów w gałęziach.\\
\begin{table}[!h]
\centering
\begin{tabular}{|y|l|l|l|}
\hline
\rowcolor{Yellow}
R [$\Omega$] & $U_{obliczone}$ [V] & $U_{zmierzone}$ [V] &$\Delta$U [V] \\
\hline
100 & 0,39 & 0,4 & 4,41 \\
\hline
200, 2000 & 0,7 & 0,68& 4,3 \\
\hline
1000 & 3,9 & 3,9 &3,86 \\
\hline
\end{tabular}
\caption{Spadki napięć na rezystorach }
\end{table}
 $\Delta$U - Różnica pomiędzy oryginalnym napięciem a tym ma rezystorze
\begin{table}[!h]
\centering
\begin{tabular}{|y|l|l|}
\hline
\rowcolor{Yellow}
R [$\Omega$] & Natężenie zmierzone [mA] &Natężenie obliczone [mA] \\
\hline
100 & 3,9 & 1,9\\
\hline
200 & 3,5 & 3,5\\
\hline
1000 & 3,7 & 1,8\\
\hline
2000 & 3,9 & 3,9\\
\hline
\end{tabular}
\caption{Prądy w gałęziach}
\end{table}
\bibliographystyle{IEEEtran}

\bibliography{IEEEabrv,refs}
Bibliografia:

http://mariusznaumowicz.ddns.net/



\end{document}
